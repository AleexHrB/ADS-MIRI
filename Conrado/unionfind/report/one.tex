\section{Introduction}
Given a non-empty set $\mathcal{A}$, a partition $\Pi$ of $\mathcal{A}$ is defined as a collection of subsets of $\mathcal{A}$, i.e., $\Pi = \{A_1, A_2, \ldots, A_k\}$, such that:

\begin{itemize}
    \item $\bigcup\limits_{i = 1}^{k} A_i = \mathcal{A}$.
    \item $A_i \cap A_j = \emptyset$ for all $i \neq j$.
\end{itemize}

The elements of $\Pi$ are called \textit{classes}. Furthermore, we define an equivalence relation between two elements $a, b \in \mathcal{A}$ by stating that $a$ and $b$ are equivalent if and only if there exists some $A_i \in \Pi$ such that $a, b \in A_i$. In this case, we say that $a$ and $b$ are equivalent and denote this by $a \equiv b$.

One can trivially verify that this binary relation is reflexive, symmetric, and transitive. These properties allow us to define a \textit{representative} for each class—an element that represents the entire class. By the reflexive, symmetric, and transitive properties, every other element in the same class is related to this \textit{representative}.

This concept, previously established by mathematicians, is widely used in Computer Science to implement the Union-Find data structure. The Union-Find data structure is designed to efficiently store and manage a partition of a set $\mathcal{A}$. It provides two fundamental operations:

\begin{itemize}
    \item \texttt{Find Operation:} Given an element $a \in \mathcal{A}$ find the representative of the class from which $a$ belongs.
    \item \texttt{Union Operation:} Given two element $a \in A_i$ and $b \in A_j$ make a union of classes $A_i$ and $A_j$. That is, given a partition $\Pi$ the result of the union operation will be a new partition $\Pi'$ such that $\Pi' =  (\Pi \setminus \{A_i,A_j\}) \cup (\{A_i \cup A_j\})$
\end{itemize}

\section{Implementation}
A C++ implementation has been made for the sake of efficiency of the Union-Find operations. A C++ class has been created for the Union-Find data structure. It consists mainly of an array in which every element will either point to another element that belongs to the same class or it will be the representative. For the representative, depending on the union strategy they will be represented differently:

\begin{itemize}
    \item With the Quick-Union strategy an element $i$ is the representative of a class if $v[i] = i$.
    \item With the other strategies (union by rank/size) they will be represented by a negative number that indicates the size/rank multiplied by $-1$ (i.e. $v[i] = -size$ or $v[i] = -rank$).
\end{itemize}

The \texttt{main.cc} file requires, as input, the size of the data structure, a Union strategy (a natural number between 0 and 2, representing Quick-Union, Union by Weight, and Union by Rank, respectively), and a Path strategy (a natural number between 0 and 3, representing No Compression, Full Compression, Path Splitting, and Path Halving). After that, the program will repeatedly request two natural numbers (the elements to be merged) and perform the corresponding union operation. This process continues until the data structure consists of a single block, at which point the program terminates.

Every $\Delta = 250$ elements, the program will output information about the current TPL and TPU of the data structure. The TPU parameter is computed using a counter that increments by one each time a pointer switch occurs, while the TPL is calculated by traversing the entire data structure. Although this approach may not be the most efficient in terms of performance, the author chose this method for computing the TPL because the focus is on obtaining the value rather than optimizing execution time. Later, execution time will be measured separately, excluding this part of the computation.

To execute the program, one can compile it using the provided \texttt{Makefile} by running \texttt{make main} and then executing the program from the command line. For example, suppose \texttt{file.txt} contains pairs of numbers (between 0 and 999) that, when processed, will lead to a single block in the Union-Find data structure. In that case, we can process these numbers using Union by Weight and Path Halving by executing the following command:

\texttt{./main.exe 1000 1 3 < file.txt}
