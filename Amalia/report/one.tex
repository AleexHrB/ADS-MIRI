\section{Introduction}  
Let $T$ be a binary tree with subtrees $T_l$ and $T_r$. We say that $T$ is a \textit{binary search tree} (BST) if it is either an empty binary tree or it contains at least one element $x$ as its root such that  

\begin{itemize}  
    \item $T_l$ and $T_r$ are also BSTs.  
    \item $\forall y \in T_l, y < x$ and $\forall z \in T_r, z > x$.  
\end{itemize}  

Although it is well known that, in the worst case, a BST behaves like a linked list (with the height of the tree being $\Theta(n)$), in this report, we focus on \textit{random BSTs} of size $n$.  

By \textit{random BSTs}, we mean the following: Given a universe of keys $U$ with $|U| = n$, we construct the BST by inserting each element of $U$ exactly once, choosing the insertion order uniformly at random.

\section{Analysis of the Average Cost of Insertions}
Let us first analyze the expected cost of inserting an element \( u \in U \) into our BST \( T \). For that, we will consider this cost as the cost of searching for \( u \) in our BST, which is valid since we assume that our search terminates in any empty subtree with identical probability. 

Let \( R(T) \) be a recurrence that, given a BST \( T \) with subtrees \( T_l \) and \( T_r \), returns the expected number of nodes that the BST will traverse during the search. Then, \( R(T) \) can be calculated as:

\[
\begin{cases}
    R(T) = 0, & \text{if } |T| = 0, \\
    R(T) = 1 + \frac{|T_l|}{n} R(T_l) + \frac{|T_r|}{n} R(T_r), & \text{otherwise}.
\end{cases}
\]

To explain \( R(T) \), it suffices to note that, since each element is drawn from a uniformly random distribution over \( n \) different values, each node has probability \( \frac{1}{n} \) of being $u$. Thus, we have:

\[
\mathbb{P}[u \in T_l] = \mathbb{P}[\exists l_i \in T_l : l_i = u] = \sum\limits_{l_i \in T_l}\mathbb{P}[l_i = u] = \sum\limits_{i = 1}^{|T_l|}\frac{1}{n} = \frac{|T_l|}{n}.
\]

Using the same reasoning, we find that \( \mathbb{P}[u \in T_r] = \frac{|T_r|}{n} \), which allows us to compute the expected number of nodes we need to traverse in order to find \( u \).

Although this recurrence is useful, we must express it in terms of the size \( n \) to obtain a general result. Let \( I_n \) be the expected cost of searching for a key \( u \) in a random BST of size \( n \), measured as the number of nodes visited until reaching \( u \). Then, \( I_n \) can be expressed as:

\[
\begin{cases}
I_n = 0, & \text{if } n = 0, \\
I_n = 1 + \frac{1}{n} \sum\limits_{k = 0}^{n-1} \left(\frac{k}{n} I_k + \frac{(n-1-k)}{n} I_{n-1-k}\right), & \text{otherwise}.
\end{cases}
\]

The key difference from the previous recurrence is that we must now consider every possible partitioning of the tree sizes as being equally likely.
