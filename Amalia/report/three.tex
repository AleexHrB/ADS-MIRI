\section{Analysis of Interleaver Insertions and Deletions}
Knuth, in Volume 3 of his well-known book \textit{The Art of Computer Programming} \cite{knuth1998art}, mentioned that random insertions and deletions indeed \textit{destroy} the randomness of a BST. This phenomenon was first observed by Gary Knott in 1972, and later, Eppinger \cite{eppinger1983empirical} demonstrated experimentally that the path length, although initially decreasing slightly, eventually increases and stabilizes after performing a quadratic number of deletions and insertions.

The following experiment aims to support the observations made by Eppinger and Knuth:
\begin{enumerate}
    \item We create a random BST of size \( n \) by generating \( n \) random keys in the interval \( [0,1] \).
    \item We perform a quadratic number of insertions and deletions as follows: we alternately insert a random element from the interval \( [0,1] \) and delete a random element from the BST.
    \item We compute the internal path length of the BST using a Breadth-First Search algorithm.
    \item We repeat all the previous steps with 20 different seeds and compute the final average.
    \item We repeat the entire experiment for different values of \( n \).
\end{enumerate}

This time, as we are doing a quadratic number of steps, because of memory and time performance I have decided to conduct this experient with Random BSTs of size $1000$ to $2000$ in increases of $50$ elements. Figure \ref{fig:plotDeletion} provides a plot of the final IPL with and without doing this alternation between deletions and insertion. As expected, we see that, indeed, the IPL of a tree increases if we perform this operations, indicating that we are unbalancing our random BST (hence, destroying our randomness).

\begin{figure}[ht]
    \centering
    \includegraphics[scale=0.65]{plotDeletion.pdf}
    \caption{Plot average IPL with and without alternating insertions and deletions}
    \label{fig:plotDeletion}
\end{figure}

\newpage

Table \ref{tab:tabDelet} provides a more numerical view of such plot. As we see, the difference in IPL gets bigger as we increases the size of $n$, definitely destroying the initial randomness that we created.

\begin{table}[ht]
    \centering
    \begin{tabular}{|c|c|c|c|}
        \hline 
        $n$ & No Deletions & Deletions & Difference \\ 
        \hline 
        1000 & 11145.85 & 11252.15 & 106.3 \\ 
        1050 & 11890.2 & 12086.95 & 196.75 \\ 
        1100 & 12627.65 & 12900.5 & 272.85 \\ 
        1150 & 13348.65 & 13680.2 & 331.55 \\ 
        1200 & 14053.475 & 14414.25 & 360.775 \\ 
        1250 & 14773.025 & 15172.7 & 399.675 \\ 
        1300 & 15557.95 & 16072.8 & 514.85 \\ 
        1350 & 16171.5 & 16617.6 & 446.1 \\ 
        1400 & 17062.5 & 17715.25 & 652.75 \\ 
        1450 & 17615.975 & 18136.7 & 520.725 \\ 
        1500 & 18530.425 & 19268.7 & 738.275 \\ 
        1550 & 19169.875 & 19860.2 & 690.325 \\ 
        1600 & 19894.3 & 20618.35 & 724.05 \\ 
        1650 & 20830.225 & 21792 & 961.775 \\ 
        1700 & 21419.875 & 22272.1 & 852.225 \\ 
        1750 & 22243.3 & 23213 & 969.7 \\ 
        1800 & 22937.55 & 23899.25 & 961.7 \\ 
        1850 & 23848.45 & 25009.05 & 1160.6 \\ 
        1900 & 24447.525 & 25495.45 & 1047.925 \\ 
        1950 & 25295.475 & 26473.65 & 1178.175 \\ 
        2000 & 26161.125 & 27483.5 & 1322.375 \\ 
        \hline 
    \end{tabular}
    \caption{Difference of IPL after doing deletions}
    \label{tab:tabDelet}
\end{table}

